\documentclass[11pt]{article}
\usepackage{header}
\def\title{HW 07}

\begin{document}
\maketitle
\fontsize{12}{15}\selectfont

\begin{center}
    Due: Saturday, 10/15, 4:00 PM \\
    Grace period until Saturday, 10/15, 6:00 PM \\
\end{center}
\normalsize
\section*{Sundry}
Before you start writing your final homework submission, state briefly how you worked on it.  Who else did you work with?  List names and email addresses.  (In case of homework party, you can just describe the group.)

\vspace{15pt}

\Question{Counting, Counting, and More Counting}

The only way to learn counting is to practice, practice, practice, so
here is your chance to do so. Although there are many subparts, each subpart is fairly short, so this problem should not take any longer than a normal CS70 homework problem. You do not need to show work, and
\textbf{Leave your answers as an expression} (rather than
trying to evaluate it to get a specific number).
\begin{Parts}

\Part How many ways are there to arrange $n$ 1s and $k$ 0s into a sequence?

\begin{solution}
  For every combination of 1s, then the 0s are already determined. Therefore, the expression is ${n + k \choose k}$
\end{solution}

\Part How many 7-digit ternary (0,1,2) bitstrings are there such that no two adjacent digits are equal?

\begin{solution}
  Consider 7 open slots, and we wish to place the numbers (0, 1, 2) into. Consider the middle one (slot 4). There are three ways of choosing which number goes there. Now consider the spots adjacent to it. Whichever number we chose for slot 4, there are 2 numbers remaining we can choose from. Then, the same logic applies the further we go out: 2 options for numbers. Therefore, there are $3 \cdot 2^6$ ways.
\end{solution}

\Part A bridge hand is obtained by selecting 13 cards from a standard
  52-card deck. The order of the cards in a bridge hand is
  irrelevant.
  \begin{enumerate}[i.]
    \item How many different 13-card bridge hands are there? 

    \begin{solution}
      This is $52 \choose 13$
    \end{solution}
    \item How many different 13-card bridge hands are there that contain no aces? 

    \begin{solution}
      We instead look at the number of hands that do contain aces, then subtract them off. We split the aces from the non-aces, so there are 4 aces and 48 non-ace cards. If there is one ace, there are $4 \choose 1$ ways of choosing that ace, along with $48 \choose 12$ cards that can go along with it, so there are ${4\choose 1}{48\choose 12}$ ways in total. If there are 2 aces, there are $4 \choose 2$ ways of doing so and a corresponding $48 \choose 11$ ways of selecting from the non aces, so ${4\choose 2}{48 \choose 11}$ ways here. Similarly, we then have ${4 \choose 3}{48 \choose 10}$ ways for 3 aces and ${4 choose 4}{48 \choose 9}$ ways for 4 aces. 

      Since these are the ways of choosing all aces, we subtract these off from the total number of bridge hands, so therefore we have 

      \[ N = {52 \choose 13} - \sum_{i = 1}^4 {4 \choose i}{48 \choose 13-i}\]
    \end{solution}
    \item How many different 13-card bridge hands are there that contain all four aces? 
    
    \begin{solution}
      We just use the result from the previous part: 

      \[ N = {4 \choose 4}{48 \choose 9} = {48 \choose 9}\]
    \end{solution}
    \item How many different 13-card bridge hands are there that contain exactly 6 spades?
    
    \begin{solution}
      There are 13 spades we can choose from, and 39 non-spade cards. Therefore, from the spade cards, we choose 6 of them, then choose the remaining 7 from the non-spade cards. Therefore, 

      \[ N = {13 \choose 6}{39 \choose 7}\]
    \end{solution}
  \end{enumerate}

\Part Two identical decks of 52 cards are mixed together, yielding a
  stack of 104 cards.
  How many different ways are there to order this stack of 104 cards?
  
\Part How many 99-bit strings are there that contain more ones than
  zeros?

  \begin{solution}
    For a given string that contains more zeroes than ones, we can flip each bit (i.e. $1 \to 0$ and $0 \to 1$) to generate a bit string with more ones than zeroes. Since there is a bijection between these two sets, we can just take the total number of bistrings and count exactly half of them. Therefore, 

    \[ N = \frac{2^{99}}{2} = 2^{98}\]
  \end{solution}
  
\Part An anagram of ALABAMA is any re-ordering of the letters of ALABAMA, i.e., any
  string made up of the letters A, L, A, B, A, M, and A, in any order.
  The anagram does not have to be an English word.
  \begin{enumerate}[i.]
    \item How many different anagrams of ALABAMA are there? 
    
    \begin{solution}
      There are 7! ways of arranging the anagram ALABAMA, but we need to account for the fact that the letter A is indistinguishable from itself. There are 4 As, so we need to divide by the 4! ways of arranging them. Therefore, 

      \[ N = \frac{7!}{4!}\]
    \end{solution}
    \item How many different anagrams of MONTANA are there?
    
    \begin{solution}
      This is the same as the last part but instead we have two letters: N and A that we need to worry about. There are 2 A's and 2 N's, so we need to divide by $2!2!$ for these two letters. Therefore, there are 

      \[ N = \frac{7!}{2! \cdot 2!}\]
    \end{solution}
  \end{enumerate}
 
\Part How many different anagrams of ABCDEF are there if:
\begin{enumerate}[i.]
  \item C is the left neighbor of E
  
  \begin{solution}
    Consider $CE$ as a single unit. Then, we are left with effectively 5 unique letters to arrange. Therefore, there are $5!$ ways of doing so. 
  \end{solution}
  \item C is on the left of E (and not necessarily E's neighbor)
  
  \begin{solution}
    Notice that of any arrangement of ABCDEF, the letter C is either on the left of E or on its right. Furthermore, if we flip the positions of only C and E, then we can arrange a bijection between the anagrams where C is on the left of E and the ones where C is on the right of E. 

    Knowing this, then this means that we can just divide the total number of ways to arrange ABCDEF by 2. Therefore, 

    \[ N = \frac{6!}{2}\]
  \end{solution}
\end{enumerate}

\Part We have 9 balls, numbered 1 through 9, and 27 bins.
  How many different ways are there to distribute these 9 balls among
  the 27 bins? Assume the bins are distinguishable (e.g., numbered 1
  through 27).

  \begin{solution}
    For the first ball, there are 27 ways of choosing which bin it goes into. The second ball has the same number of choices, and so on and so forth. Therefore, the total number of ways is $27^4$.
  \end{solution}
  
\Part How many different ways are there to throw 9 identical balls
  into 27 bins? Assume the bins are distinguishable (e.g., numbered 1
  through 27).

  \begin{solution}
    We use the balls and bins formula here. We have 27 bins and 9 balls, which is identical to having 9 stars and 26 bars. Therefore, 

    \[ N = {35 \choose 9}\] 
  \end{solution}
 
\Part We throw 9 identical balls into 7 bins.
  How many different ways are there to distribute these 9 balls among
  the 7 bins such that no bin is empty? Assume the bins are
  distinguishable (e.g., numbered 1 through 7). 

  \begin{solution}
    If no bin is empty, then it must contain at least 1 ball. Therefore, this problem is equivalent to asking how many ways are there to place $9 - 7 = 2$ balls into one of 7 bins, where they are distinbuishable. There are 7 choices for where to place the first ball, and there are also 7 choices for where we could place the second ball. Therefore, there are 

    \[ N = 7^2 = 49\] 

    total ways.
  
  \end{solution}

\Part There are exactly 20 students currently enrolled in a class.
  How many different ways are there to pair up the 20 students, so
  that each student is paired with one other student? Solve this in at least 2 different ways. \textbf{Your final answer must consist of two different expressions. }


  \begin{solution}
    There are ${20 \choose 2}$ ways of choosing the first pair, then ${18 \choose 2}$ ways for the second pair, then so on and so forth. Therefore, there are 

    \[ N = \sum_{n = 0}^{5} {20 - 2n \choose 2}\]

    ways. 

    Alternatively, we can consider these 20 students as a graph of 20 nodes, and we are asked to pair 
  \end{solution}

  
\Part How many solutions does $x_0 + x_1 + \cdots + x_k = n$ have, if each $x$ must be a non-negative integer?

\begin{solution}
  We can imagine each of $x_0, x_1, \dots x_k$ as a bin, and we wish to place $n$ balls into these bins. This means that we have $k+1$ bins and $n$ balls, which translates to $n$ balls and $k$ bars. Using stars and bars, we have:

  \[ N = {n + k \choose n}\]
\end{solution}

\Part How many solutions does $x_0 + x_1 = n$ have, if each $x$ must be a \emph{strictly positive} integer?

\begin{solution}
  The value of $x_0$ ranges from $1$ to $n-1$, so there are $n-1$ solutions. 
\end{solution}

\Part How many solutions does $x_0 + x_1 + \cdots + x_k = n$ have, if each $x$ must be a \emph{strictly positive} integer?

\begin{solution}
  If each $x$ must be strictly positive, then it has a value of at least 1. Therefore, this is the same situation as if we subtract 1 from each of $x_0, x_1, \dots, x_k$ and then consider the sum total to be $n-k$ instead, but each of $x_0, x_1, \dots, x_k$. Therefore, we can use part (l) to solve this problem. Therefore, we have $n-k$ balls and $k+1$ bins, which corresponds to $n-k$ balls and $k$ bars, so therefore we have: 

  \[ N = {n \choose k}\]
\end{solution}

\end{Parts}
\pagebreak
\Question{Grids and Trees!}

Suppose we are given an $n \times n$ grid, for $n \geq 1$, where one starts at $(0,0)$ and goes to $(n,n)$. On this grid, we are only allowed to move left, right, up, or down by increments of $1$.

{\color{blue}{The shortest path that exists is a path where we only either go up or to the right, since if we ever move either left or down, then we'd have to retrace our steps at some point, thus generating a longer path.}}
\begin{Parts}

\Part How many shortest paths are there that go from $(0,0)$ to $(n,n)$?

\begin{solution}
  This is equivalent to choosing $n$ steps up and $n$ steps to the right, so we get $2n \choose n$
\end{solution}

\Part How many shortest paths are there that go from $(0,0)$ to $(n-1,n+1)$?

\begin{solution}
  Similarly to the previous problem, this time we choose where to place $n-1$ steps to the right, so this is $2n \choose {n-1}$
\end{solution}

\end{Parts}

Now, consider shortest paths that meet the conditions where we can only visit points $(x, y)$ where $y \leq x$. That is, the path cannot cross
line $y = x$. We call these paths $n$-legal paths for a maze of side length $n$. Let $F_n$ be the number of $n$-legal paths.
\begin{ResumeParts}
\Part Compute the number of shortest paths from $(0, 0)$ to $(n, n)$ that cross $y = x$. (Hint: Let $(i, i)$ be the first time the shortest path crosses the line $y = x$. Then the remaining path starts from $(i,i+1)$ and continues to $(n,n)$. If in the remainder of the path 
one exchanges $y$-direction moves with $x$-direction moves and
vice versa, where does one end up?)

\begin{solution}
  Following the hint, let $(i, i)$ denote the first time the shortest path crosses the line $y = x$. The remaining path starts from $(i, i+1)$ and continues to $(n,n)$. Note that once this occurs, any path that travels from $(i, i+1)$ automatically crosses $y = x$. 

  Starting from $(i, i+1)$ to $(n, n)$, we can think of this as the same as walking from $(0, 0)$ to $(n - i, n - i - 1)$, for which there are ${2(n-i) - 1 \choose {n - i}}$ ways of doing so. Therefore, the total number of paths that cross $y = x$ just means that we need to sum over all $i$, where $i$ ranges from $1$ to $n - 1$, since $(n -1, n)$ is the last time our path can cross $y =x$: 

  \[ N = \sum_{i = 1}^{n - 1} {2 (n -i) - 1 \choose {n - i}}\]
\end{solution}

\Part Compute the number of shortest paths from $(0, 0)$ to $(n, n)$ that do not cross $y = x$. (You may find your answers from parts (a) and (c) useful.)

\begin{solution}
  We know that $2n \choose n$ is the total number of ways from $(0, 0)$ to $(n, n)$ and from part (b) we've solved for the number of paths that do cross $y = x$, so taking of the difference of the two will give us the paths that do not cross $y = x$: 

  \[ N = {2n \choose n} - \sum_{i = 1}^{n - 1} {2 (n -i) - 1 \choose {n - i}}\]
\end{solution}

\Part A different idea is to derive a recursive formula for the number of paths. Fix some $i$ with $0 \leq i \leq n - 1$. We wish to count the number of $n$-legal paths where the last time the path touches the line $y = x$ is the point $(i, i)$. Show that the number of such paths is $F_i \cdot F_{n-i-1}$. (Hint: If $i=0$, what are your first and last moves, and where is the remainder of the path allowed to go?)

\begin{solution}
  Suppose the last time the path touches $y = x$ is at $(i, i)$. Firstly, there are $F_i$ ways of reaching $(i, i)$, since by definition $F_i$ only counts legal paths. Now, once we reach $(i, i)$, then the only way we can move is to the right, since moving to $(i, i+1)$ would result in an illegal path. Furthermore, the last move must go upwards, since we can only approach $(n, n)$ from $(n, n-1)$. Therefore, the remainder of the path walks from $(i+1, i)$ to $(n, n-1)$. 

  Now notice the following: the number of times we need to travel up is $n - (i + 1) = n - i - 1$, and the number of ways we need to travel to the righis $n - 1 - i$, so we actually have a square of length $n - i - 1$! The number of legal ways of performing this traversal is $F_{n - i - 1}$.

  Since for every path in $F_n$ we can choose $F_{n - i - 1}$ ways of continuing the path from $(i, i)$ to $(n , n)$, then the total number of ways will be the product of the two. Therefore, 

  \[ N = F_i \cdot F_{n - i - 1}\]

  
\end{solution}

\Part Explain why $F_n = \sum_{i = 0}^{n - 1} F_i \cdot F_{n - i - 1}$.

\begin{solution}
  The first time our path can touch $y = x$ is at $(0, 0)$, corresopnding to $i = 0$. The last time a path can tough $y =x$ is at $(i -1, i-1)$, so therefore in order to count all the $n$-legal paths, we need to sum over all of $i = 0$ to $i= n-1$ to count all $n$-legal paths. All these paths are distinct from one another, becuase for each $i$, it is assumed that $(i, i)$ is the last time the path crosses $y =x$, and thus we can simply sum over all $i$.
  % Building off the previous part, there could be multiple points that could be marked as the ``last time'' our path touches the line $y = x$. These points are $(0, 0), (1, 1), \dots, (n -1, n-1)$, since reaching $(n,n)$ as the last time is already counted in 
\end{solution}

\Part Create and explain a recursive formula for the number of trees with $n$ vertices $(n \geq 1)$, where each non-root node has degree at most $3$, and the root node has degree at most $2$. Two trees are different if and only if either left-subtree is
different or right-subtree is different.

(Notice something about your formula and the grid problem. Neat!)

\begin{solution}
  Notice that if every non root node has degree at most 3, then this means that this is a binary tree. Starting from the root node, the number of trees is the sum of the number of trees in the left subtree and the number in the right subtree. 

  Let $F_n$ denote the number of ways of counting the number of trees with $n$ vertices. This number is equal to the product of the number of trees in the two trees directly beneath it, since for every tree on the left we choose, there are $F_k$ ways to choose the right, and vice versa. If the tree on the left has $k$ vertices, then the tree on the right has $n-k-1$ vertices, since we are excluding the root node. Therefore for a given $k$, 

  \[ F_n = F_{n-k-1} \cdot F_k\]

  Now note that we can have up to $k = 1$ up to $k = n -2$ nodes on either side of the tree, so we need to sum over all valid $k$. Therefore, our final sum is:

  \[ F_n = \sum_{k = 1}^{n - 2} F_{n-k-1} \cdot F_k\]
  
\end{solution}

\end{ResumeParts}
\pagebreak
\Question{Fermat's Wristband}

Let $p$ be a prime number and let $k$ be a positive integer.
We have beads of
$k$ different colors, where any two beads of the same color are indistinguishable.

\begin{Parts}
    \Part
    We place $p$ beads onto a string.
    How many different ways are there to construct such a sequence of $p$ beads with up to $k$ different colors?

    \begin{solution}
      There are $k$ colors to choose from for every bead, so there are $k^p$ ways in total of constructing.
    \end{solution}

    \Part How many sequences of $p$ beads on the string are there that use at least two colors?

    \begin{solution}
      Consider the following: there are only $k$ ways of placing the beads on the string such that it only uses a single color (1 way per color), so therefore the number of ways of counting two colors would be the same as taking the total number and subtracting off these $k$ ways. Therefore, 

      \[ N = k^p - k = k(k^{p-1} - 1)\]
      % Consider this string of $p$ beads. Choose a single color for the first bead, meaning that we are left with $p - 1$ remaining beads and $p$ colors to choose from. 
    \end{solution}

    \Part
    Now we tie the two ends of the string together, forming a
    wristband.
    Two wristbands are equivalent if the sequence of colors on one
    can be obtained by rotating the beads on the other.
    (For instance, if we have $k=3$ colors, red (R), green (G), and
    blue (B), then the length $p = 5$ necklaces RGGBG, GGBGR, GBGRG, BGRGG, and GRGGB are all
    equivalent, because these are all rotated versions of each other.)

    How many non-equivalent wristbands are there now?
    Again, the $p$
    beads must not all have the same color.
    (Your answer should be a simple function of $k$ and $p$.)

    [\textit{Hint}: Think about the fact that rotating all the beads on the wristband to another
        position produces an identical wristband.]


      \begin{solution}
        The number of ways to place beads such that they don't all have the same color was solved in the previous section: there are $k^p - k$ ways. Now, we must get rid of roatations of these wristbands. Notice that for any arrangement of $p$ beads, we can perform $p$ rotations, all of which result in the same wristband. Since this can be done for every single arrangement of beads we create, we must divide our result by $p$ to account for this. Therefore, 

        \[ N = \frac{k^p - k}{p}\]
      \end{solution}

    \Part Use your answer to part (c) to prove Fermat's little theorem.


    \begin{solution}
      We know that the total number of ways is given by 

      \[ N = \frac{k^p - k}{p} = \frac{k(k^{p-1} - 1)}{p}\]

      First of all, this number must be an integer, since it makes no sense to say that there are 1.5 ways of creating a wristband. Therefore, the fraction must always be divisible by $p$. This means that either $k$ is divisible by $p$ or $k^{p-1} - 1$ is divisible by $p$. 

      Now suppose that $k$ and $p$ are coprime, and thus they do not share any factors (just like in FLT). Then, the fact that $N$ must be an integer means that $\frac{k^{p-1} - 1}{p}$ must be an integer, or more formally in terms of modular arithmetic, 

      \[ k^{p-1} - 1 \equiv 0 \pmod{p}\]

      or equivalently, 

      \[ k^{p-1} \equiv 1 \pmod{p}\] 

      which is exactly Fermat's little theorem, given $p$ is prime and $k$ is coprime to $p$. 
    \end{solution}
\end{Parts}
\pagebreak
\Question{Counting on Graphs + Symmetry}

\begin{Parts}

    \Part How many ways are there to color the faces of a cube using exactly $6$ colors, such that each face has a different color? Note: two colorings are considered the same if one can be obtained from the other by rotating the cube in any way.

    \begin{solution}
      First, there are 6! ways of coloring the cube using exactly 6 colors. Now we will look at how many ways we are overcounting, by considering how many ways we can rotate the cube for a specific coloring. 

      Suppose the cube is centered at $(0, 0, 0)$ in 3d space, and we are looking at the cube from the $+z$ axis. There are 6 ways to orient the cube such that we are looking at a unique face each time without changing the coloring, so we need to divide by 6 to account for this. 

      Furthermore, for any given face, there are four ways we could rotate the cube along the $x-y$ plane without changing the coloring (1 for every face), so therefore we need to divide by a further 4. to accout for this as well. Therefore, the total number of colorings is:
      % Consider a specific cube coloring. There are 6! ways of coloring the cube, given 6 colors. Now, consider the number of ways we could orient the cube such that we are facing a particular face: there are 6 distinct faces, so we need to divide our total counting by 6. Further, there are 4 ways of rotating the cube while mainining the same face pointing towards us, so we divide by a further 4. Thus, the total number of ways is 

      \[ N = \frac{6!}{6 \cdot 4}\]
    \end{solution}
    

    \Part How many ways are there to color a bracelet with $n$ beads using $n$ colors, such that each bead has a different color? Note: two colorings are considered the same if one of them can be obtained by rotating the other.

    \begin{solution}
      There are $n!$ ways of arranging $n$ beads using $n$ colors. To see this, imagine an ordered pair $(c_1, c_2, \dots, c_n)$ that denotes an arrangement. Within each ordered pair, there are $n$ ways of rotating it such that the coloring is preserved (by the question), so therefore we need to divide by $n$ to account for our overcounting. Therefore, the total number of ways is

      \[ N = \frac{n!}{n} = (n-1)!\] 
    \end{solution}
    

    \Part How many distinct undirected graphs are there with $n$ labeled vertices? Assume that there can be at most one edge between any two vertices, and there are no edges from a vertex to itself. The graphs do not have to be connected.
    
    \begin{solution}

    \end{solution}

    \Part How many distinct cycles are there in a complete graph $K_n$ with $n$
     vertices? Assume that cycles cannot have duplicated edges. Two cycles are
     considered the same if they are rotations or inversions of each other (e.g.
     $(v_1,v_2,v_3,v_1)$, $(v_2,v_3,v_1,v_2)$ and $(v_1,v_3,v_2,v_1)$ all count as
     the same cycle).
     
     \begin{solution}
      We use the fact that a cycle can be represented as a string of $k$ vertices, as described in the problem statement. Since the graph is complete, every vertex has a path to every other vertex. This means that for a cycle to exist, we can essentially just choose strings of length $k$. The number of ways of choosing strings of length $k$ is $n \choose k$. 
      
      Next, consider a specific cycle. For a cycle of length $k$, there are $k!$ ways of arranging the vertices in a certain order, since they are distinct. However, we need to now get rid of rotations. Let the cycle be denoted by $(v_1, v_2, \dots, v_k)$. For every permutation of $v_1, \dots, v_k$, there are $k$ ways of rotating this permutation, so we need to divide by $k$ to exclude them from our counting. Therefore, there are $\frac{k!}{k} = (k-1)!$ ways of arranging the vertices. 

      Furthermore, we need to get rid of inversions. If a cycle is defined as $(v_1, v_2, \dots, v_k)$, then its inversion is defined as $(v_k, v_{k-1}, \dots, v_1)$. This means that in our current counting method, each cycle currently contributes two cycles rather than one, so therefore to remove them we divide our counting by 2. Therefore, the total number of allowed cycles: 

      \[ N = \sum_{k = 3}^n{n \choose k} \frac{(k-1)!}{2}\] 
      \end{solution}

\end{Parts}
\pagebreak
\Question{Proofs of the Combinatorial Variety}

Prove each of the following identities using a combinatorial proof.

\begin{Parts}

\Part For every positive integer $n>1,$ 
\[\sum_{k=0}^n k \cdot \binom{n}{k} = n\cdot \sum_{k=0}^{n - 1}\binom{n - 1}{k}.\]

\begin{solution}
  Firstly, notice that on the left hand side, $k = 0$ corresponds to 0, so insetad we try to prove 

  \[ \sum_{k=1}^n k \cdot \binom{n}{k} = n\cdot \sum_{k=0}^{n - 1}\binom{n - 1}{k}\] 

  instead. First, consider the right hand side, where we have an arrangement of $n-1$ slots, and we fill in $k$ of them. Here, there are $n -1 \choose k$ ways of filling in these $k$ spots. Now, if we add in an $n$-th spot, there are ${n \choose n -1} = n$ ways of choosing where this empty spot lies, so therefore for every $k$ there are $n{n-1 \choose k}$ ways of selecting $k$ slots. 

  Notice how there's also an alternate way to phrase this choosing operation. This is the same as the following: for every $k$, we choose $k+1$ slots to fill in, then choose one to remove, leaving us with $k$ filled slots. Therefore, this counting can also be re-expressed as $k {n \choose k+1}$. 

  Finally, notice how we are summing the left and right term. For every $k$ on the right hand side, the corresponding value is 1 larger on the left. In other words, for every operation with $k$ we are performing on the right hand side, we are doing the same on the left with $k+1$ objects, and becuase we've argued earlier that they are in fact the same operation, the left and right hand sides are equal. $\blacksquare$
\end{solution}

\Part For each positive integer $m$ and each positive integer $n > m,$
\[\sum_{a + b + c = m} \binom{n}{a}\cdot\binom{n}{b}\cdot\binom{n}{c} = \binom{3n}{m}.\]
(Notation: the sum on the left is taken over all triples of nonnegative integers $(a,b,c)$ such that $a + b + c = m.$)

\begin{solution}
  Consider the left hand side of this equation. If we imagine a line of $n$ slots, the left hand side of the expression is essentially asking us for the number of ways to select $a, b, c$ slots from $n$ such that $a + b + c = n$, and that the same slot can be selected up to three times. Since the same slot can be chosen across $a, b$, and $c$, then the total number of ways to do so for a given $a, b, c$ is to take the product of the three. We them sum over all ways of choosing $a, b, c$ such that $a+b+c = n$ to get the total number of ways.

  Now consider the right hand side: imagine a line of $3n$ slots, which we can label in the following manner: label the first $n$ as $1, 2, \dots, n$, then label them again from $1, 2, \dots, n$, then repeat the same process with the third set of $n$ slots. Now, we select $m$ of these $3n$ slots. But notice that since $a + b + c = m$, this is the same as choosing $a$ slots from the first $n$ slots, then choosing $b$ out of the next $n$ slots, and finally $c$ out of the final $n$ slots (notice that the slot with the same number can be selected up to three times, just like the left hand side). Since these selection processes are independent, then we take the product of these three, then sum over all ways of choosing $a, b, c$ such that $a+b+c = n$, which is exactly the equation on the left hand side. $\blacksquare$
\end{solution}
\end{Parts}

\Question{Fibonacci Fashion}

You have $n$ accessories in your wardrobe, and you'd like to plan which ones to wear each day for the next $t$ days. As a student of the Elegant Etiquette Charm School, you know it isn't fashionable to wear the same accessories multiple days in a row. (Note that the same goes for clothing items in general).
Therefore, you'd like to plan which accessories to wear each day represented by subsets $S_1,S_2,\ldots,S_t$, where $S_1 \subseteq \{1,2,\ldots,n\}$ and for $2\leq i \leq t$, $S_i \subseteq\{1,2,\ldots,n\}$ and $S_i$ is disjoint from $S_{i-1}$. 

\begin{Parts}
\Part For $t\geq 1$, prove that there are $F_{t+2}$ binary strings of length $t$ with no consecutive zeros (assume the Fibonacci sequence starts with $F_0=0$ and $F_1=1$).

\begin{solution}
  We proceed by induction. For $t = 1$, there are 2 binary strings: 1, and 0, which do not have conescutive zeroes. Since $F_3 = 2$, then the base case is proven. 

  \textbf{Inductive Step: } Suppose for all $k \le n$ that for a binary string of length $k$, there are $F_{k+2}$ ways of arranging with no consecutive zeroes. 

  Now we aim to prove this is true for a binary string of length $k+1$. We know that a binary string of length $k+1$ can be deconstructed into a binary string of length $k$ and a singular bit. For the string of length $k$, we know from our inductive hypothesis that there are $F_{k+2}$ ways of arranging those bits.
  
  Now we consider that the $k+1$th bit is a 1. Therefore, this means that this bit can go anywhere within our sequence of $k$ while still preserving the fact that we have no consecutive zeroes. 
\end{solution}

\Part Use a combinatorial proof to prove the following identity, which, for $t\geq 1$ and $n\geq 0$, gives the number of ways you can create subsets of your $n$ accessories for the next $t$ days such that no accessory is worn two days in a row:
\[\sum_{x_1 \geq 0} \sum_{x_2\geq 0}\cdots \sum_{x_t \geq 0} {n \choose x_1}{n-x_1 \choose x_2}{n-x_2 \choose x_3}\cdots {n-x_{t-1}\choose x_t}= (F_{t+2})^n.\]
(You may assume that $\binom{a}{b} = 0$ whenever $a < b$.)

\end{Parts}

\end{document}
