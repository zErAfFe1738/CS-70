\documentclass[11pt]{article}
\usepackage{header}
\def\title{HW 01}

\begin{document}
\maketitle
\fontsize{12}{15}\selectfont

\begin{center}
    Due: Saturday, 9/3, 4:00 PM \\
    Grace period until Saturday, 9/3, 6:00 PM \\
\end{center}

\section*{Sundry}
Before you start writing your final homework submission, state briefly how you worked on it.  Who else did you work with?  List names and email addresses.  (In case of homework party, you can just describe the group.)

{\color{blue}{I did not work with anyobdy to complete this homework. To calculate large factorials (such as $100!$), I used WolframAlpha to get an approximate value for its decimal expansion, but aside from that I did not use any other online source to complete this homework.}}

\vspace{15pt}

\Question{Solving a System of Equations}

Alice wants to buy apples, beets, and carrots. An apple, a beet, and a carrot cost 16 dollars, two apples and three beets cost 23 dollars, and one apple, two beets, and three carrots cost 35 dollars. What are the prices for an apple, for a beet, and for a carrot, respectively? Set up a system of equations and show your work.

\begin{solution}
    Let $x$ be the number of apples, $y$ be the number of beets and $z$ be the number of carrots. Then from the information in the question, we have: 

    \begin{align}
        \label{eq1} x + y + z &= 16\\
        \label{eq2} 2x + 3y &= 23\\
        \label{eq3} x + 2y + 3z &= 35
    \end{align}

    
\end{solution}

\Question{Calculus Review}

\begin{Parts}
    \Part Compute the following integral:
        \[
            \int_0^{\infty} \sin(t)e^{-t} \dd{t}.
        \]
    
    \begin{solution}
        Call the integral $I$. Doing integration by parts gives:

           \[ I = -\sin(t)e^{-t}\bigg\rvert_0^\infty + \int_0^\infty \cos(t)e^{-t}\dd{t}\]

           Note that the first part vanishes at $\infty$ since $e^{-\infty} = 0$, a it vanishes at $0$ since $\sin(0) = 0$. Thus we're left with:

           \[ \int_0^\infty \cos(t) e^{-t} \dd t\]

           Now we do integration by parts again: 


            \begin{align*}
                I &= -\cos(t)e^{-t} \bigg\rvert_0^\infty - \int_0^\infty \sin(t)e^{-t} \dd t\\
                &= [0 + \cos(0) e^0] - \int_0^\infty \sin(t) e^{-t} \dd t
            \end{align*}

            Notice that the integral on the right is the same as our original integral to solve. Thus we can write:

            \begin{align*}
                2I &= 1\\
                \therefore I &= \frac{1}{2}
            \end{align*}

            Thus

            \[ \int_0^\infty \sin(t) e^{-t} \dd t = \frac{1}{2}\]
    \end{solution}
    \Part Find the minimum value of the following function over the reals and determine where it occurs.
    \[f(x) = \int_{0}^{x^2} e^{-t^2} \dd{t}.\]
    Show your work.

    \begin{solution}
        It's clear that the function $e^{-t^2}$ is positive for all real $t$, and if we interpret the integral as the area under the curve, it's clear that the integral over any nonzero interval $[0, x^2]$ will result in a positive result. Thus, the minimum of $f(x)$ is 0 when $x = 0$.
    \end{solution}

    \Part Compute the double integral
    \[\iint_{R} 2x + y \dd{A},\]
    where $R$ is the region bounded by the lines $x = 1$, $y = 0$, and $y = x$.

    \begin{solution}
    The line $y = x$ intersects the line $x = 1$ at $(1, 1)$. Therefore, we can set the bounds as follows: 

    \[ \int_0^1\int_0^x 2x + y \ \dd y \ \dd x\]

    Solving: 

    \begin{align*}
        \int_0^1\int_0^x 2x + y \ \dd y \ \dd x &= \int_0^1 (2xy + \frac{y^2}{2})\bigg\rvert_0^x \dd x\\
        &= \int_0^1 2x^2 + \frac{x^2}{2}\\
        &= \left(\frac{2}{3}x^3 + \frac{x^3}{6}\right)\bigg\rvert_0^1\\
        &= \frac{1}{2}
    \end{align*}

    \end{solution}
\end{Parts}

\Question{Implication}
Which of the following assertions are true no matter what proposition $Q$ represents? For any false assertion, state a counterexample (i.e. come up with a statement $Q(x, y)$ that would make the implication false). For any true assertion, give a brief explanation for why it is true.

\begin{Parts}

\item
$\exists x \exists y Q(x,y) \implies \exists y \exists x Q(x,y)$.

\begin{solution}
    This is true. $\exists x \exists y \equiv \exists y \exists x$
\end{solution}
    

\item
$\forall x \exists y Q(x,y) \implies \exists y\forall x Q(x,y)$.
    
\begin{solution}
    This is not true. The statement on the left indicates that regardless of any $x$ we choose we can choose a corresponding $y$. On the other hand, the statement on the right hand side indicates that there is a $y$ such that $Q(x, y)$ is true regardless of what value of $x$ we choose. These two statements are not equivalent, since choosing $x$ and a corresponding $y$ such that $Q(x, y)$ is true does not guarantee that there exists such a $y$ that makes $Q(x, y)$ true for all $x$.
\end{solution}

\item
$\exists x \forall y Q(x,y) \implies \forall y \exists x Q(x,y)$.

\begin{solution}
    This is not true. This statement is not true for the same reason why part (b) is not true. 
\end{solution}
    

\item
$\exists x \exists y Q(x,y) \implies \forall y \exists x Q(x,y)$.

\begin{solution}
    This is not true. The statement on the left hand side indicates the existence of $x$ and $y$ such that $Q(x, y)$ is true, but it makes no statement on the fact that there is a corresponding $x$ regardless of what $y$ we choose. 
\end{solution}
    

\end{Parts}

\Question{Logical Equivalence?}

Decide whether each of the following logical equivalences is correct and justify your answer. 

\begin{Parts}
    \Part $\forall x \; \bigl( P(x) \wedge Q(x) \bigr)~\equiv~\forall x \; P(x) \wedge \forall x \; Q(x)$
    
    \Part $\forall x \; \bigl( P(x) \vee Q(x) \bigr)~\equiv~\forall x \; P(x) \vee \forall x \; Q(x)$
    
    \Part $\exists x \; \bigl( P(x) \vee Q(x) \bigr)~\equiv~\exists x \; P(x) \vee \exists x \; Q(x)$
    
    \Part $\exists x \; \bigl( P(x) \wedge Q(x) \bigr)~\equiv~\exists x \; P(x) \wedge \exists x \; Q(x)$
    
\end{Parts}

\Question{Preserving Set Operations}

For a function $f$, define the image of a set $X$ to be the set $f(X) = \{y~|~y = f(x) \text{ for some } x \in X\}$. Define the inverse image or preimage of a set $Y$ to be the set $f^{-1}(Y) = \{x~|~f(x) \in Y\}$. Prove the following statements, in which $A$ and $B$ are sets. By doing so, you will show that inverse images preserve set operations, but images typically do not.

\textit{Recall: For sets $X$ and $Y$, $X=Y$ if and only if $X \subseteq Y \text{ and } Y \subseteq X$. To prove that $X \subseteq Y$, it is sufficient to show that $(\forall x)~((x \in X) \implies (x \in Y))$.}

\begin{Parts}
    \Part $f^{-1}(A \cap B) = f^{-1}(A) \cap f^{-1}(B)$.
    \Part $f^{-1}(A \setminus B) = f^{-1}(A) \setminus f^{-1}(B)$.
    \Part $f(A \cap B) \subseteq f(A) \cap f(B)$, and give an example where equality does not hold.
    \Part $f(A \setminus B) \supseteq f(A) \setminus f(B)$, and give an example where equality does not hold.
\end{Parts}

\Question{Prove or Disprove}
For each of the following, either prove the statement, or disprove by finding a counterexample.
\begin{Parts}
	\Part $(\forall n \in \mathbb{N})$ if $n$ is odd then $n^2 + 4n$ is odd.

    \begin{solution}
        We can show this statement true by factoring $n^2 + 4n = n(n+4)$. Since both $n$ and $n + 4$ are odd numbers, and the product of two odd numbers is always odd, then $n^2 + 4n$ is always odd given odd $n$.
    \end{solution}

	\Part $(\forall a, b \in \mathbb{R})$ if $a + b \le 15$ then $a \le 11$ or $b \le 4$.

    \begin{solution}
        Consider the case where $a > 11$ and $b > 4$, then $a + b > 11 + 4 = 15$, which is a contradiction to $a + b \le 15$.
    \end{solution}
	\Part $(\forall r \in \mathbb{R})$ if $r^2$ is irrational, then $r$ is irrational.
\begin{solution}
    Suppose that $r^2$ is irrational and $r$ is rational. This means that $r =\frac{p}{q}$ for $p, q \in \mathbb Z$. Thus, $r^2 = \frac{p^2}{q^2}$, which is clearly a rational number, so $r$ must also be an irrational number.
\end{solution} 
	\Part $(\forall n \in \mathbb{Z}^+)$ $5n^3 > n!$. (Note: $\mathbb{Z}^+$ is the set of positive integers)
    \begin{solution}
    This is clearly false, $5(100)^3 = 5 \times 10^6$ is much smaller than $100! \sim 10^{157}$. (this factorial was computed using WolframAlpha)
    \end{solution}
\end{Parts}

\Question{Rationals and Irrationals}
Prove that the product of a non-zero rational number and an irrational number is irrational.

\begin{solution}
    Let $a \in \mathbb Q$ and $b \notin \mathbb Q$. We are then asked to prove that $ab \notin \mathbb Q$. We prove this by contradiction.

    Suppose that $ab \in \mathbb Q$. Then we can write:

    \[ ab = \frac{p}{q} \cdot a = \frac{r}{s}\]

    Where $p, q, r, s \in \mathbb Z$ to denote rationals. Then, we can divide both sides by $\frac{p}{q}$:

    \[ b = \frac{r}{s} \cdot \frac{q}{p} = \frac{rq}{sp}\] 

    Where the right hand side represents a rational number. But since $b$ is an irrational number, we have reached a contradiction. Thus, the product of a nonzero rational number and an irrational number is irrational.
\end{solution}

\Question{Twin Primes}

\begin{Parts}

\Part
Let $p > 3$ be a prime. Prove that $p$ is of the form $3k + 1$ or $3k-1$ for some integer $k$.

\begin{solution} 
    We can rephrase this question slightly: that every prime number $p$ is either 1 or 2 modulo 3. This statemt is clearly true, since all numbers which are $0$ modulo 3 are clearly divisible by 3, and thus cannot be prime. As a result, all primes are either 1 or 2 modulo 3. 
\end{solution} 

\Part
\textit{Twin primes} are pairs of prime numbers $p$ and $q$ that have a difference of 2. Use part (a) to prove that 5 is the only prime number that takes part in two different twin prime pairs.

\begin{solution}
   From part $a$ we know that every prime can be written as $3k + 1$ or $3k - 1$. We work through both cases and show that this is not possible:

   \begin{itemize}
        \item \textit{Case 1: $p = 3k + 1$.} If $p = 3k + 1$, then $p - 2 = 3k - 1$ and $p + 2 = 3k+3 = 3(k+1)$, which is clearly not prime.
        \item \textit{Case 2: $p = 3k-1$.} If $p = 3k-1$, then $p - 2 = 3k - 3 = 3(k-1)$ which is clearly not prime, so there cannot be three consecutive twin primes.
   \end{itemize}

   The only situation where we do have a number taking part in two different twin prime pairs is if $3k - 3 = 3$, which gives $k = 2$ and thus $p = 5$, the only number that satisfies this condition.
\end{solution}

\end{Parts}

\end{document}
